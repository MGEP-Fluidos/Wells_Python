% !TEX root =technical_dossier.tex

% ACTUATOR DISK FORMULATION ON A MONOPLANE ISOLATED WELLS TURBINE
\section*{Actuator-disk formulation on a monoplane isolated Wells turbine}{\label{sec:actuator-disk_formulation_monoplane_isolated_Wells_turbine}}
\noindent The actuator-disk formulation is meant to account for the radial variation of the flow both upstream and downstream a given turbine stage. For such a purpose, the formulation considers that the turbine stage can be represented by an infinitely thin disk that induces a jump on the energetic variables of the fluid, without affecting the continuity of the flow across it. Empirical results show that the departure from the radial equilibrium condition is not negligible, requiring to introduce such an effect somehow. On this respect, the actuator-disk formulation is the simplest way of modelling the radial variation of the flow both upstream and downstream the stage.\\
As opposed to the two-dimensional BEM theory, the axial component of the velocity is not necessarily constant in an actuator-disk formulation, neither in the axial nor in the radial directions upstream and downstream the disk representing the turbine stage. The straightforward manner to consider the redistribution of mass-flow is to apply the conservation of the axial momentum and equate it to the force exerted by the flow on the actuator-disk:
\begin{equation}{\label{eq:axial_momentum_conservation_disk}}
	f_{x,i}=\dot{m_{i}}\left(v_{x1,i}-v_{x2,i}\right)=\rho_{a}v_{xd,i}A_{ann,i}\left(v_{x1,i}-v_{x2,i}\right) \ ,
\end{equation}
where $\dot{m_{i}}$ refers to the differential mass flow-rate passing through a blade element, and $A_{ann,i}$ stands for the differential area filled by such an element. The $v_{xd,i}$ addresses the velocity at the disk that corresponds to the blade element itself.\\
To compute such a velocity, it suffices with applying the conservation of energy equating the axial power acting on the disk (\Cref{eq:energy_conservation_axial_power}) and the power lost by the flow (\Cref{eq:energy_conservation_lost_power}):
\begin{equation}{\label{eq:energy_conservation_axial_power}}
	P_{x,i}=f_{x,i}v_{xd,i}=\dot{m_{i}}\left(v_{x1,i}-v_{x2,i}\right)v_{xd,i}=\rho_{a}v_{xd,i}^{2}A_{ann,i}\left(v_{x1,i}-v_{x2,i}\right) \ ,
\end{equation}
\begin{equation}{\label{eq:energy_conservation_lost_power}}
	P_{x,i}=\frac{1}{2}\dot{m_{i}}\left(v_{x1,i}^{2}-v_{x2,i}^{2}\right)=\frac{1}{2}\rho_{a}v_{xd,i}A_{ann,i}\left(v_{x1,i}^{2}-v_{x2,i}\right) \ .
\end{equation}
Thus, the matching of \Cref{eq:energy_conservation_axial_power,eq:energy_conservation_lost_power} means that the kinetic energy extracted from the flow corresponds to the energy absorbed by the disk. It follows, from those expressions, that the velocity at the disk can be expressed as:
\begin{equation}{\label{eq:disk_velocity}}
	v_{xd,i}=\frac{1}{2}\left(v_{x1,i}+v_{x2,i}\right) \ .
\end{equation}
The implementation of the actuator-disk formulation introduces yet another iterative procedure in the code, which encompasses the two-dimensional BEM theory within. The workflow of the iteration procedure goes as follows:
\begin{itemize}
	\item{The two-dimensional BEM theory is applied for resolving the velocity triangles and energetic outcomes of the turbine stage.}
	\item{The velocity at the disk is computed.}
	\item{The Reynolds numbers and load coefficients are recomputed based on the velocity at the disk.}
	\item{The two-dimensional BEM theory is reapplied for resolving the velocity triangles and energetic outcomes.}
	\item{An energy checking is performed on the outcome power. In case it is not fulfilled, the disk velocity is re-computed, which closes the cycle.}
\end{itemize}
\begin{figure}[t!]
	\includegraphics{figures/actuator-disk_pseudocode_snippet.pdf}
	\caption{Pseudocode snippet corresponding to the actuator-disk formulation.}
	\label{fig:actuator-disk_pseudocode_snippet}
\end{figure}
After achieving energy convergence, the velocity triangles that correspond to the blade elements of the considered turbine stage have been resolved under both the radial equilibrium condition and the actuator-disk condition. The pseudocode snippet showing the actuator-disk formulation is provided in \Cref{fig:actuator-disk_pseudocode_snippet}.
